\documentclass[a4paper,3p,sort&compress]{elsarticle}

\usepackage[draft]{hyperref}
\usepackage{url}
\usepackage{booktabs}
\usepackage{graphicx}
\usepackage{xspace}
\usepackage{booktabs}
\usepackage[draft,inline,nomargin]{fixme}
\usepackage{makecell}
\usepackage{lineno}
\usepackage{natbib}
\usepackage{amsmath}
\DeclareRobustCommand{\citeext}[1]{\citeauthor{#1}~\cite{#1}}

\journal{-}

%% `Elsevier LaTeX' style
\bibliographystyle{plain}
%%%%%%%%%%%%%%%%%%%%%%%

\begin{document}
\linenumbers

% Macro para escribir NO$_2$
\newcommand{\no}{NO\textsubscript{2}\xspace}

\begin{frontmatter}

  \title{Dicrete uncertainty visualization for no levels time series forecast}


  \author{Sebasti\'an P\'erez Vasseur}
  \author{Jos\'e L. Aznarte}
  \address{Artificial Intelligence Department\\Universidad Nacional de
    Educaci\'on a Distancia --- UNED\\c/ Juan del Rosal, 16, Madrid, Spain}
  \ead{jlaznarte@dia.uned.es}
  

\begin{abstract}
  
\end{abstract}

\begin{keyword}
probabilistic forecasting \sep visualization \sep dotplot
\end{keyword}

\end{frontmatter}

%\linenumbers

\section{Introduction}
\label{sec:intro}

Hothorn \emph{et al.} stated that the real
objective in a machine learning regression analysis is to find the full conditional distribution
of the target variable. Currently, many machine learning models today only provide a point distribution, which is usually the predicted
mean of the target variable. Then the users of the model must take a decision based on the value predicted. However, only point information without uncertainty creates a false sense of deterministic result that can lead to false confidence in the decision. That's why as noted by Fernandez et Al., uncertainty improves decision making. However, users rely on visual tools and the typical visualization tools are not suited for probabilistic information, those visual tools are adapted to deterministic information.

There have been several workarounds around such issue. The first approach is to complement the point information with uncertainty. The first example that comes to our mind is the error bars in the bar chart. However, Kai et Al. showed that the information has to be intrinsic to the visualization and offered better alternatives. 

The PDF function provides a full picture of the probability. Same goes for the CDF function. However, as Fernandez et Al. states, the lack of statistical knowledge of some users deter from using those functions.  
Users can rely on less data to understand the uncertainty: Stripplots provide deciles of the target variable and the box plot provides the mean and the quantiles. This is the reason why they are more used than the pdf or cdf functions.

Also Kai et Al. investigation noted that users prefer to think in terms of natural frecuencies (2/10) instead of probabilities (.2), and therefore it's better to use discrete outcomes to communicate probabilities. cite1 cite2 used a novel visualization chart called quantile dotplot and they obtained better interpretation results using this chart to communicate uncertainty.

Quantile dotplots are built from the PDF as we can see in figure .. Then instead of using area to calculate the probability, we count the number of dots in the desired range.

\section{Time Series}
\label{sec:time_series}

Time series data can also have uncertainty and they have usually been graphed in 3 main ways: gradient, interval and EPS.
Gradient uses color to communicate the percentile, however as noted by (get ref) color is a bad visual encoding to show value as users have problems to see the value it is encoded.
Interval only shows the low and bottom quantile (usually 5\% and 95\%) and lacks information that the user would have to imagine. For symwtrical pdf, it can be useful, but non symwtrical are not shown.
Finally, EPS diagrams show the range and some quantile as a box plot would do.

All those visualizations have a problem, they either encode with a bad encoding the values or have missing information quantiles. 
Many times the objective is to calculate the probability that the value will be in a certain range for a certain hour.
We propose a novel visualization technique to show the Data, that is as well discrete. We will call it Time series dotplots . I'm time series dotplots, we build a stripplot for each hour, but then we use a dot instead of a bar. This way we can count the probability of a range, simply by counting the number of dots in that range at that hour.


\subsection{Application: No levels forecast in Madrid}

Air quality has become a concern in recent years and lately pollution peaks have forced local authorities to take measures like cutting traffic or blocking certain city areas. Those measures are purely reactive and therefore a forecast of pollution might be preferrable as proactive measures could be taken and prevent the pollution peak before it happens.

\section{Results}
\label{sec:results}

\section{Conclusions}
\label{sec:concl}

\bibliography{refs}

\end{document} 

